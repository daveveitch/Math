\documentclass{article}
\usepackage[english]{babel}
\usepackage{etoolbox}
\usepackage{pgfplots}
\usetikzlibrary{arrows}
\usepackage[utf8]{inputenc}
\usepackage[margin=0.5in]{geometry}
\usepackage{amsmath,amssymb}
\usepackage{graphicx}
\usepackage{mathtools}
\usepackage[makeroom]{cancel}
\graphicspath{ {./images/} }
\renewcommand{\arraystretch}{3}
\setlength\parindent{0pt}
\newcommand*{\Perm}[2]{{}^{#1}\!P_{#2}}%
\newcommand*{\Comb}[2]{{}^{#1}C_{#2}}%



\title{Probability Question - Discontinuities of a Distribution Function  }
\author{David Veitch}
\date{January 2020 (Updated December 2021)}

\begin{document}

\maketitle

\section{Question}

Show that a distribution function has at most countably many
discontinuities.

\section{Answer}

Let $F$ be an arbitrary distribution function and $\mathcal{D} =\{x \; |\; \exists \; \text{a discontinuity at } F(x)\}$. Define $F(x-)=\lim_{y\nearrow x} F(y)$ and $F(x+)=\lim_{y\searrow  x}F(y)$\\

Since $F$ is a distribution function it is non-decreasing. For an arbitrary $x_1 \in \mathcal{D}$, since a discontinuity exists at $F(x_1)$, we know that $F(x_1-)<F(x_1+)$. Let $A_1$ be the interval $\big(F(x_1-),F(x_1+)\big)$. Notice that $A_1$ is an open interval in $[0,1]$ therefore it must contain a rational number $q_1$ (see Section \ref{intervalproof} for a proof of this). \\

Therefore for any discontinuity $x_i$ there $\; \exists \;$ an open interval $A_i$, which is disjoint from all $A_j, j\neq i$, which contains a rational number $q_i$. Therefore we can create a one-to-one function $f:\mathcal{D}\rightarrow \mathbb{Q} \Rightarrow |\mathcal{D}| \leq |\mathbb{Q}| \Rightarrow \mathcal{D}$ is countable $\Rightarrow$ a distribution function has at most countably many discontinuities.

\section{Proof - In Any Open Interval $(a,b)$ where $0\leq a<b\leq 1$ There Exists a Rational Number in $(a,b)$}\label{intervalproof}

This is true because if $a_0.a_1a_2a_3\dots$ is the infinite decimal expansion of $a$, and $b_0.b_1b_2b_3\dots$ is the infinite decimal expansion of $b$ there will exist some smallest index $i \in \mathbb{Z}^+\cup \{0\}$ such that $0<10^{-(i+1)} \leq b-a \leq 10^{-(i)}$.\\

$\Rightarrow a_0.a_1\dots a_ia_{i+1}a_{i+2} + 10^{-(i+2)} \in \mathbb{Q}$ and $a_0.a_1\dots a_ia_{i+1}a_{i+2} + 10^{-(i+2)} \in (a,b)$.

\section{Supplemental Material}

This question is Exercise 1.2.3. in the book Probability: Theory and Examples by Rick Durrett.\\

I like this problem for its simplicity, but also because the result is very useful. Over the Graduate Probability course I took in the Fall of 2019 it came up again and again.

\end{document}
