\documentclass{article}
\usepackage[english]{babel}
\usepackage{etoolbox}
\usepackage{pgfplots}
\usetikzlibrary{arrows}
\usepackage[utf8]{inputenc}
\usepackage[margin=0.5in]{geometry}
\usepackage{amsmath,amssymb}
\usepackage{graphicx}
\usepackage{mathtools}
\usepackage[makeroom]{cancel}
\graphicspath{ {./images/} }
\renewcommand{\arraystretch}{3}
\setlength\parindent{0pt}
\newcommand*{\Perm}[2]{{}^{#1}\!P_{#2}}%
\newcommand*{\Comb}[2]{{}^{#1}C_{#2}}%



\title{Topology Question - Continuous $f:\mathbb{R}_\text{co-countable} \longrightarrow \mathbb{R}_\text{usual}$ }
\author{David Veitch}
\date{August 2019}

\begin{document}

\maketitle

\section{Question}

Prove that there exist no continuous function $f:\mathbb{R}_\text{co-countable} \longrightarrow \mathbb{R}_\text{usual}$.\\

\textbf{Recall:}
\begin{itemize}
    \item $\mathbb{R}_\text{co-countable}$ and $\mathbb{R}_\text{usual}$ are $\mathbb{R}$ with the co-countable ($\mathbb{R}$, $\mathbb{R}_\text{co-countable}$) and usual topologies ($\mathbb{R}$, $\mathbb{R}_\text{usual}$) respectively  
    \item In ($\mathbb{R}$, $\mathbb{R}_\text{co-countable}$), $\mathbb{R}_\text{co-countable} \coloneqq \; \{U \subseteq \mathbb{R} \; | \; \mathbb{R} \setminus U \text{ is countable} \} \cup \{\emptyset\}$
    \item In ($\mathbb{R}$, $\mathbb{R}_\text{usual}$), $\mathbb{R}_\text{usual}$ is generated by the basis $\mathcal{B} \coloneqq \{(a,b) \; | \; a,b \in \mathbb{R}, \; a < b \}$
\end{itemize}

\section{Answer}

Suppose there did exist a continuous function $f:\mathbb{R}_\text{co-countable} \longrightarrow \mathbb{R}_\text{usual}$. Therefore $\forall \; (a,b) \in \mathbb{R}$, we have by the continuity of $f$ that $f^{-1}\big( (a,b) \big) \in \mathbb{R}_\text{co-countable}$.\\

Let $D_{1}$ and $D_{2}$ be disjoint open intervals in $\mathbb{R}$ (and hence $D_{1},  D_{2} \in \mathbb{R}_\text{usual}$).\\

Since $f$ is continuous, $f^{-1}(D_1), f^{-1}(D_2) \in \mathbb{R}_\text{co-countable}$. Therefore by the definition of $\mathbb{R}_\text{co-countable}$, $\mathbb{R} \setminus f^{-1}(D_1)$ is countable, and  $\mathbb{R} \setminus f^{-1}(D_2)$ is countable.\\

Also, note that $f^{-1}(\mathbb{R} \setminus D_1) = f^{-1}(\mathbb{R}) \setminus f^{-1}(D_1)$, and $f^{-1}(\mathbb{R} \setminus D_2) = f^{-1}(\mathbb{R}) \setminus f^{-1}(D_2)$.\\

Also, $D_1 \subseteq \mathbb{R} \setminus D_2 \Rightarrow f^{-1}(D_1) \subseteq f^{-1}(\mathbb{R} \setminus D_2)$. Given  $f^{-1}(\mathbb{R} \setminus D_2)$ is countable $\Rightarrow$ $f^{-1}(D_1)$ is countable.\\

However, we also know that $\mathbb{R}=f^{-1}(D_1) \cup \big( \mathbb{R}\setminus f^{-1}(D_1) \big)$. Therefore $\mathbb{R}$ is a union of two countable sets $\Rightarrow \mathbb{R}$ is countable. This is obviously a contradiction. Therefore there exists no continuous function $f:\mathbb{R}_\text{co-countable} \longrightarrow \mathbb{R}_\text{usual}$.


\section{Supplemental Material}

This is not the most difficult problem in topology I came across, however I think it is nice because of the contradiction that is ultimately arrived at, that $\mathbb{R}$ is countable.\\

Special thanks to Ivan Khatchatourian who assigned a version of this question as a `Big List' homework question for MAT327 Introduction to Topology at the University of Toronto in the summer of 2019.

\end{document}
